\documentclass[11pt,a4paper]{article}
\usepackage[utf8]{inputenc}
\usepackage[margin=1in]{geometry}
\usepackage{amsmath}
\usepackage{amsfonts}
\usepackage{amssymb}
\usepackage{enumitem}
\usepackage{hyperref}
\usepackage{fancyhdr}

\pagestyle{fancy}
\fancyhf{}
\fancyhead[L]{How to train the Neural Web}
\fancyhead[R]{\thepage}

\title{\textbf{How to train the Neural Web}}
\author{Witold Warchoł}
\date{}

\begin{document}

\maketitle

\section{Input Data Files Required}

\subsection{Text Processing Data}
\begin{itemize}[leftmargin=0.5in]
    \item \texttt{vocabulary.txt} - Word vocabulary for text processing and 
    embedding generation. Each line contains one word or token.
    
    \item \texttt{custom\_embeddings.txt} - Pre-trained word embeddings. 
    Format: word followed by space-separated float values representing the 
    embedding vector.
    
    \item Text input string - Training text data provided as string constant 
    or loaded from file. Example: "Apple, banana, cherry, date, and 
    elderberry are fruits."
\end{itemize}

\subsection{System State Files}
\begin{itemize}[leftmargin=0.5in]
    \item \texttt{memory\_system.dat} - Binary file containing previous 
    memory system state (optional for fresh training)
    
    \item \texttt{hierarchical\_memory.dat} - Hierarchical memory structure 
    data (optional)
    
    \item \texttt{system\_parameters.dat} - Optimization parameters and 
    performance metrics from previous runs
    
    \item \texttt{metacontroller.dat} - Meta-controller state for adaptive 
    learning
    
    \item \texttt{motivation.dat} - Intrinsic motivation system parameters
    
    \item \texttt{performance\_metrics.dat} - Network performance tracking 
    data
    
    \item \texttt{reflection\_params.dat} - Self-reflection system 
    parameters
    
    \item \texttt{identity\_system.dat} - Self-identity framework state
    
    \item \texttt{knowledge\_filter.dat} - Knowledge categorization system
    
    \item \texttt{metacognition.dat} - Metacognitive processing parameters
    
    \item \texttt{meta\_learning.dat} - Meta-learning state and strategies
\end{itemize}

\section{User-Defined Training Parameters}

\subsection{Network Training Configuration}
\begin{itemize}[leftmargin=0.5in]
    \item Initial learning rate (typically 0.01-0.1)
    \item Number of training steps/epochs
    \item Batch size for processing optimization
    \item Target output vectors for supervised learning components
    \item Performance thresholds for adaptation triggers
\end{itemize}

\subsection{Memory System Configuration}
\begin{itemize}[leftmargin=0.5in]
    \item Memory decay rates for temporal forgetting
    \item Importance thresholds for memory consolidation
    \item Similarity thresholds for memory clustering
    \item Capacity limits for each memory tier (short/medium/long-term)
\end{itemize}

\subsection{Ethical Framework Parameters}
\begin{itemize}[leftmargin=0.5in]
    \item Ethical principle definitions and weights
    \item Decision evaluation criteria
    \item Moral compass confidence thresholds
    \item Ethical constraint enforcement levels
\end{itemize}

\section{Training Data Generation}

\subsection{Task Prompts}
The system requires task-specific prompts and verification criteria:
\begin{itemize}[leftmargin=0.5in]
    \item Task descriptions for each training step
    \item Verification instructions with expected outcomes
    \item Confidence scoring mechanisms
    \item Reasoning validation frameworks
\end{itemize}

\subsection{Target Generation}
\begin{itemize}[leftmargin=0.5in]
    \item Define target output generation strategy
    \item Specify error calculation methods
    \item Set performance evaluation metrics
    \item Configure adaptive target adjustment mechanisms
\end{itemize}

\section{Social and Emotional Training Data}

\subsection{Social Interaction Scenarios}
\begin{itemize}[leftmargin=0.5in]
    \item Behavioral patterns for person modeling
    \item Negotiation contexts and outcome preferences
    \item Empathy development scenarios
    \item Social feedback examples
\end{itemize}

\subsection{Emotional State Definitions}
\begin{itemize}[leftmargin=0.5in]
    \item Emotional trigger patterns
    \item Regulation strategies and thresholds
    \item Cognitive-emotional integration parameters
    \item Emotional memory importance weighting
\end{itemize}

\section{Imagination and Creativity Parameters}

\subsection{Scenario Generation}
\begin{itemize}[leftmargin=0.5in]
    \item Divergence factor ranges for creative exploration
    \item Plausibility evaluation criteria
    \item Scenario blending coefficients
    \item Creativity-coherence balance parameters
\end{itemize}

\subsection{Problem-Solving Configuration}
\begin{itemize}[leftmargin=0.5in]
    \item Error thresholds triggering imagination activation
    \item Solution influence weights
    \item Exploration vs exploitation balance
    \item Creative solution evaluation metrics
\end{itemize}

\section{Training Pipeline Configuration}

\subsection{Initialization Strategy}
\begin{enumerate}
    \item Load existing system states or initialize with default values
    \item Configure dynamic parameter adaptation rates
    \item Set up performance tracking and optimization schedules
    \item Initialize ethical and social processing frameworks
\end{enumerate}

\subsection{Training Loop Requirements}
\begin{enumerate}
    \item Define input tensor generation from text and memory
    \item Configure predictive coding and error computation
    \item Set adaptation trigger frequencies (memory consolidation, 
    parameter optimization, system reflection)
    \item Establish performance evaluation intervals
\end{enumerate}

\subsection{Web Search Integration}
\begin{itemize}[leftmargin=0.5in]
    \item Search query generation strategies
    \item Result filtering and relevance scoring
    \item Knowledge integration confidence thresholds
    \item External information validation criteria
\end{itemize}

\section{Performance Optimization Requirements}

\subsection{Dynamic Parameter Tuning}
\begin{itemize}[leftmargin=0.5in]
    \item Learning rate adaptation schedules
    \item Batch size optimization criteria
    \item Network plasticity adjustment parameters
    \item Noise tolerance configuration
\end{itemize}

\subsection{System Health Monitoring}
\begin{itemize}[leftmargin=0.5in]
    \item Error rate thresholds for intervention
    \item Memory usage optimization targets
    \item Identity consistency verification schedules
    \item Security validation parameters
\end{itemize}

\section{Output and Evaluation Configuration}

\subsection{Performance Metrics Definition}
\begin{itemize}[leftmargin=0.5in]
    \item Loss function selection and weighting
    \item Convergence criteria and stopping conditions
    \item Progress visualization and reporting intervals
    \item System introspection and self-questioning protocols
\end{itemize}

\subsection{State Persistence}
\begin{itemize}[leftmargin=0.5in]
    \item Save intervals for system components
    \item Backup and recovery strategies
    \item State validation and consistency checking
    \item Performance history tracking and analysis
\end{itemize}

\section{Training Methodology}

\subsection{Pre-Training Setup}
\begin{enumerate}
    \item Prepare all required input files in correct formats
    \item Initialize system parameters with conservative values
    \item Set up performance monitoring and logging systems
    \item Configure backup and state persistence mechanisms
\end{enumerate}

\subsection{Training Phases}

\subsubsection{Phase 1: Basic Network Stabilization (Steps 1-100)}
\begin{itemize}[leftmargin=0.5in]
    \item Focus on basic neuron state convergence
    \item Use simple target outputs and low learning rates (0.001-0.01)
    \item Monitor network stability and prevent oscillations
    \item Establish baseline memory formation patterns
    \item Disable complex subsystems (imagination, social processing)
\end{itemize}

\subsubsection{Phase 2: Memory System Integration (Steps 101-300)}
\begin{itemize}[leftmargin=0.5in]
    \item Gradually increase memory system engagement
    \item Introduce hierarchical memory consolidation
    \item Begin predictive coding integration
    \item Monitor memory-neuron interaction stability
    \item Adjust memory decay and importance parameters
\end{itemize}

\subsubsection{Phase 3: Multi-System Activation (Steps 301-600)}
\begin{itemize}[leftmargin=0.5in]
    \item Enable emotional processing subsystem
    \item Activate imagination system with low creativity factors
    \item Introduce basic social modeling components
    \item Begin ethical framework evaluation
    \item Monitor inter-system interference patterns
\end{itemize}

\subsubsection{Phase 4: Advanced Integration (Steps 601-1000)}
\begin{itemize}[leftmargin=0.5in]
    \item Full system activation with dynamic adaptation
    \item Enable web search integration and external knowledge
    \item Activate creative problem-solving mechanisms
    \item Implement complete ethical decision-making
    \item Focus on emergent behavior optimization
\end{itemize}

\subsection{Training Loop Implementation}

Each training step follows this sequence:
\begin{enumerate}
    \item \textbf{Input Generation}: Create input tensor from text, memory, 
    and context
    \item \textbf{Forward Processing}: Update neuron states through network
    \item \textbf{Memory Operations}: Retrieve relevant memories and update 
    working memory
    \item \textbf{Subsystem Processing}: Apply emotional, social, ethical, 
    and imaginative processing
    \item \textbf{Error Computation}: Calculate prediction errors and loss 
    functions
    \item \textbf{Backpropagation}: Update weights and connection strengths
    \item \textbf{Parameter Adaptation}: Adjust learning rates and system 
    parameters
    \item \textbf{State Consolidation}: Update memory systems and save 
    critical states
    \item \textbf{Performance Evaluation}: Assess progress and trigger 
    optimizations
\end{enumerate}

\subsection{Training Monitoring}

\subsubsection{Critical Metrics to Track}
\begin{itemize}[leftmargin=0.5in]
    \item Loss convergence and stability
    \item Memory system utilization and effectiveness
    \item Inter-system communication quality
    \item Ethical alignment maintenance
    \item Identity consistency over time
    \item Creative output quality and relevance
\end{itemize}

\subsubsection{Warning Signs Requiring Intervention}
\begin{itemize}[leftmargin=0.5in]
    \item Oscillating or diverging loss functions
    \item Memory system overflow or corruption
    \item Identity consistency failures
    \item Ethical constraint violations
    \item Excessive system resource consumption
    \item Degraded performance after optimization attempts
\end{itemize}

\subsection{Adaptive Training Strategies}

\subsubsection{Dynamic Parameter Adjustment}
\begin{itemize}[leftmargin=0.5in]
    \item Reduce learning rate when loss plateaus
    \item Increase exploration when performance stagnates
    \item Adjust memory consolidation frequency based on utilization
    \item Modify creativity factors based on problem-solving success
    \item Scale ethical constraint enforcement with system maturity
\end{itemize}

\subsubsection{Curriculum Learning Approach}
\begin{itemize}[leftmargin=0.5in]
    \item Start with simple, well-defined tasks
    \item Gradually increase task complexity and ambiguity
    \item Introduce social scenarios after basic competence
    \item Add ethical dilemmas only after stable decision-making
    \item Incorporate creative challenges as final training phase
\end{itemize}

\subsection{Troubleshooting Common Issues}

\subsubsection{Training Instabilities}
\begin{itemize}[leftmargin=0.5in]
    \item \textbf{Oscillating outputs}: Reduce learning rate, increase 
    damping factors
    \item \textbf{Memory overflow}: Adjust consolidation frequency, 
    increase decay rates
    \item \textbf{System conflicts}: Temporarily disable conflicting 
    subsystems, retrain integration
    \item \textbf{Identity drift}: Strengthen identity verification, 
    restore from backup
\end{itemize}

\subsubsection{Performance Degradation}
\begin{itemize}[leftmargin=0.5in]
    \item \textbf{Loss plateaus}: Increase learning rate temporarily, 
    add exploration noise
    \item \textbf{Overfitting}: Introduce regularization, reduce 
    model complexity
    \item \textbf{Memory saturation}: Clear low-importance memories, 
    optimize storage
    \item \textbf{Ethical violations}: Strengthen constraint enforcement, 
    retrain ethical framework
\end{itemize}

\subsection{Training Completion Criteria}

Training is considered successful when:
\begin{enumerate}
    \item Loss function converges to acceptable levels
    \item Memory system operates efficiently without overflow
    \item All subsystems interact harmoniously
    \item Ethical decision-making remains consistent
    \item Identity system maintains coherence over time
    \item Creative outputs demonstrate both novelty and relevance
    \item Performance metrics stabilize across evaluation periods
\end{enumerate}

\subsection{Post-Training Validation}

\subsubsection{System Testing Protocol}
\begin{enumerate}
    \item Present novel scenarios requiring multi-system integration
    \item Evaluate ethical decision consistency under pressure
    \item Test memory recall and consolidation effectiveness
    \item Assess creative problem-solving capabilities
    \item Verify identity stability across diverse contexts
    \item Confirm social interaction appropriateness
\end{enumerate}

Success requires careful tuning of interaction parameters between subsystems 
and continuous monitoring of emergent behaviors to ensure stable and 
beneficial learning outcomes.

\end{document}
